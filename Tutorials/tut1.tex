\documentclass[11pt]{article}
\usepackage[english]{babel}
\usepackage{minted}
\usepackage{graphicx}
\usepackage[left=25mm, top=25mm, bottom=30mm, right=25mm]{geometry}
\usepackage[colorlinks=true, linkcolor=blue, urlcolor=cyan]{hyperref}

\title{COL351: Tutorial 1}
\author{Sayam Sethi}
\date{August 2021}

\begin{document}

\maketitle

\tableofcontents

\section{Question 1}
The time complexity can be formulated as $T(n) = T(\sqrt{n}) + T(n - \sqrt{n}) + c \cdot n$. On building the recursion tree, we observe that each level takes $O(n)$ steps. Thus, the time complexity can be reduced to:
\begin{equation}
    T(n) = O(n) \cdot H(n), H(n) = 1 + H(n - \sqrt{n})
\end{equation}
It can be shown that $H(n) = \Theta(\sqrt{n})$ as follows:

\subsection{$H(n) = o(\sqrt{n})$}
In each step, the reduction in the value of $n$ is $\leq \sqrt{n}$. Thus, the total number of steps are $\geq n / \sqrt{n} = \sqrt{n}$. Therefore, $H(n) = o(\sqrt{n})$.

\subsection{$H(n) = O(\sqrt{n})$}
Consider the equation $H(n) = k + O(H(n/2))$. Here, $k$ represents the steps taken for $n$ to reduce to a value of $n/2$. We know that the decrement of $n$ for these $k$ steps, lies between $\sqrt{n}$ and $\sqrt{n/2}$. Thus, $k \leq \frac{n - n/2}{\sqrt{n/2}} = \sqrt{n/2}$.\par
Now, reducing the above equation, we get the summation, $H(n) = \sqrt{n/2} + \sqrt{n/4} + \cdots + 1$. This summation is $O(\sqrt{n})$. Thus, $H(n) = O(n)$.

\section{Question 2}


\end{document}

